\documentclass{article}

% Packages
\usepackage{amsmath} %i.e. \begin{equation*} Ecuacion \end{equation*} para ecuaciones no numeradas
\usepackage{graphicx}
\usepackage{subcaption} % para grupos de figuras
\usepackage{amssymb} % mas simbolos
\usepackage{algorithmicx}
\usepackage{algpseudocode}

\title{Borrador Obligatorio MetNum 2019}
\date{03-09-19}
\author{Autores}

\begin{document}
    \pagenumbering{gobble}
    \maketitle
    \newpage
    \pagenumbering{arabic}

    %Ej1
    \section{Minimo de una funcion en $\mathbb{R}^2$}

    %Ej1.1
    \subsection{Graficar la funcion}

    %Ej1.2
    \subsection{Hallar $Q$ y $b$}
    Comentario: Aca se muestra el desarrollo de como se llega a $Q$ y $b$.
    Sean $Q \in \mathbb{R}^{2\times2}$ y $b \in \mathbb{R}^{2\times1}$ tales que

    \begin{equation*}
        Q = \begin{pmatrix} q_{11} & q_{12} \\ q_{21} & q_{22} \end{pmatrix}
    \end{equation*}

    \begin{equation*}
        b = \begin{pmatrix} b_1 \\ b_2 \end{pmatrix}
    \end{equation*}

    
    Sea $f(z) = (z^{T}Qz - 2b^{T}z) + 2e^{x+y}$ con $z = (x,y)^T \in \mathbb{R}^{2\times1}$.
    Desarrollando esta expresion se llega a
    $f(x,y) = q_{11}x^2 + q_{22}y^2 + (q_{12} + q_{21})xy - 2b_1x - 2b_2y + 2e^{x+y}$

    Para que esta expresion sea igual a (1), se tiene que
    $q_{11} = 1$ $q_{12 + q_{21} = -\frac{2}{3}}$ $q_{22} = 3$ $b_1 = -1$ $b_2 = 2$
    
    Como se vera mas adelante resulta conveniente elegir $q_{12} = q_{21} = -\frac{1}{3}$.

    De esta forma, se tiene que
    $Q =  \begin{pmatrix} \phantom{-}1 & -\frac{1}{3} \\ -\frac{1}{3} & \phantom{-}3\end{pmatrix} $
    y
    $b = \begin{pmatrix} -1\\ \phantom{-}2 \end{pmatrix}$

    \begin{equation}
        f(z) = (z^{T}Qz - 2b^{T}z) + 2e^{x+y}, z = (x,y) \in \mathbb{R}^{2\times1}
    \end{equation}

    %Ej1.3
    \subsection{Expresion matricial para puntos criticos}
    Comentario: Aca se muestra el desarrollo de como se llega a expresar la ec de puntos criticos en forma matricial.
    
    Para resolver este ejercicio, se calcula el gradiente de $f$ y se desarrolla la expresion de $F(z)$, mostrando que se
    llega a la misma expresion.

    El gradiente de $f$ igualado a cero da el sistema de ecuaciones:
    
    \begin{equation}
        \nabla f = \left( \frac{\partial f}{\partial x}, \frac{\partial f}{\partial y} \right)
        =
        \left( 2x - \frac{2}{3}y + 2 + 2e^{x+y}, 6y - \frac{2}{3}x - 4 + 2e^{x+y} \right) = (0,0)
    \end{equation}

    Que se puede simplificar dividiendo ambas expresiones entre 2.
    
    \begin{equation}
        \nabla f = \left( \frac{\partial f}{\partial x}, \frac{\partial f}{\partial y} \right)
        =
        \left( x - \frac{1}{3}y + 1 + e^{x+y}, 3y - \frac{1}{3}x - 2 + e^{x+y} \right) = (0,0)
    \end{equation}

    Por otro lado, se desarrolla la expresion de $F(z)$, que queda de la misma forma que el sistema de ecuaciones anterior.
    \begin{align}
        F(z) &= (Qz - b) + \left(\begin{matrix} 1\\1 \end{matrix}\right)e^{x+y}
        \\
        F(z) &= \begin{pmatrix} \phantom{-}1 & -\frac{1}{3} \\ -\frac{1}{3} & \phantom{-}3\end{pmatrix} \begin{pmatrix} x\\y \end{pmatrix}
                -\begin{pmatrix} -1\\\phantom{-}2 \end{pmatrix} + \begin{pmatrix} e^{x+y}\\e^{x+y} \end{pmatrix}
        \\
        F(z) &= \begin{pmatrix} \phantom{-}x -\frac{1}{3}y + 1 +e^{x+y} \\ -\frac{1}{3}x + 3y - 2 +e^{x+y} \end{pmatrix} = \begin{pmatrix} 0\\0 \end{pmatrix}
    \end{align}

    %Ej1.4
    \subsection{Hallar a mano los puntos criticos}
    Aca se muestra alguna imagen y una tabla de valores x,y f y F = grad(f).

    %Ej1.5
    \subsection{Puntos criticos; un primer metodo}
    Hallar los puntos criticos de $f(z)$ equivale a hallar los ceros de su gradiente que, como se vio en 1.3, es la
    solucion del sistema de ecuaciones $F(z) = 0$.
    Esto se puede hacer mediante la funcion fsolve de octave:
        fsolve(@"nombre de la funcion",$X_0$ = punto de partida)
    Utilizando como punto de partida a (1, 1), se obtiene como punto critico a (-1.28035, 0.38786).
    Siendo este el minimo mostrado en 1.4.

    %Ej2
    \section{Newton Raphson}
    Segundo problema

    %Ej2.1
    \subsection{Describir metodo de Newton-Raphson}

    %Ej2.2
    \subsection{]Matriz jacobiana de la funcion $F(z)$}
    Al igual que en la parte 1.3, empezamos por calcular la matriz jacobiana.
    
    Recordamos que la expresion de la funcion F es:
    \begin{equation*}
        F(x,y) = \left( 2x - \frac{2}{3}y + 2 + 2e^{x+y}, 6y - \frac{2}{3}x - 4 + 2e^{x+y} \right)
    \end{equation*}

    Entonces, su jacobiana es:
    \begin{equation}
        \mathbb{J}_F(x,y) =
        \begin{pmatrix}
            \frac{\partial F_1}{\partial x} & \frac{\partial F_1}{\partial y} \\
            \frac{\partial F_2}{\partial x} & \frac{\partial F_2}{\partial y}
        \end{pmatrix}
        =
        \begin{pmatrix}
            \phantom{-}1 + e^{x+y} & -\frac{1}{3} + e^{x+y} \\
            -\frac{1}{3} + e^{x+y} & \phantom{-}3 + e^{x+y}
        \end{pmatrix}
    \end{equation}

    Se quiere probar que $\mathbb{J}_F(x,y) = Q + aa^T$, con $a(x,y) = \begin{pmatrix} 1\\1 \end{pmatrix}\sqrt{e^{x+y}} \in \mathbb{R}^{2 \times 1}$
    es una expresion equivalente.
    
    Desarrollamos la expresion:
    \begin{equation}
        \mathbb{J}_F(x,y) = 
        \begin{pmatrix}
            \phantom{-}1 & -\frac{1}{3} \\
            -\frac{1}{3} & \phantom{-}3
        \end{pmatrix}
        +
        \begin{pmatrix} 1 & 1 \\ 1 & 1 \end{pmatrix} \sqrt{e^{x+y}}
        =
        \begin{pmatrix}
            \phantom{-}1 + e^{x+y} & -\frac{1}{3} + e^{x+y} \\
            -\frac{1}{3} + e^{x+y} & \phantom{-}3 + e^{x+y}
        \end{pmatrix}
    \end{equation}
    Se ve entonces que efectivamente se llega a lo mismo.

    %Ej2.3
    \subsection{Implementacion de Newton-Raphson en $\mathbb{R}^2$}
    Se Implementa el metodo de Newton-Raphson para resolver el sistema $F(z) = 0$
    \begin{algorithm}[H]
        \caption{Newton-Raphson}\label{NR1}
        \small
        \centering
        \begin{algorithmic}[1]
        \For{k = 0 to iters do}
            \State{$z^{(k+1)} \leftarrow z^{(k)} - J^{-1}_F(z^{(k)}).F(z^{(k)})$}
        \EndFor \\
        \Return $z^{(k)}$
        \end{algorithmic}
    \end{algorithm}
    En Octave, el calculo de $J^{-1}_F(z^{(k)}).F(z^{(k)})$ se realiza con el operador "\"
    %Ej2.4
    \subsection{}

    %Ej2.5
    \subsection{}

    %Ej2.6
    \subsection{}

    %Ej3
    \section{Cholesky}
    Tercer problema

    %Ej3.1
    \subsection{}

    %Ej3.2
    \subsection{}

    %Ej3.3
    \subsection{}

    %Ej3.4
    \subsection{}

    %Ej4
    \section{Minimo de una Funcion en $R^n$}
    Cuarto problema

    %Ej4.1
    \subsection{}

    %Ej4.2
    \subsection{}

    %Ej4.3
    \subsection{Implementacion de Cholesky en una matriz tridiagonal simetrica.}
    \begin{equation*}
        Q = \begin{pmatrix} 
                \alpha_1 & \beta_1 &    0   & \cdots  &   0     \\ 
                \beta_1  & \ddots  & \ddots & \ddots  & \vdots  \\
                  0      & \ddots  & \ddots & \ddots  &   0     \\
                \vdots   & \ddots  & \ddots & \ddots  & \beta_n \\
                  0      & \cdots  &    0   & \beta_n & \alpha_n
            \end{pmatrix}
    \end{equation*}

    \begin{equation*}
        L_{n} = \begin{pmatrix} 
                1   &        &         &         &   \\ 
                l_1 &   1    &         &         &   \\
                    & \ddots & \ddots  &         &   \\
                    &        & l_{n-2} &    1    &   \\
                    &        &         & l_{n-1} & 1
            \end{pmatrix}
    \end{equation*}

    \begin{equation*}
        \Delta_n = \begin{pmatrix} 
                \delta_1 &        &          \\ 
                         & \ddots &          \\
                         &        & \delta_n
            \end{pmatrix}
    \end{equation*}

    con \begin{equation} l_i = \frac{\beta_i}{\delta_i} \end{equation}
    y \begin{equation} l_i = \frac{\beta_i}{\delta_i} \end{equation}

    \begin{equation}
        \delta_i =
            \begin{cases}
                \alpha_1 \text{ si } i = 1 \\
                \alpha_i - \frac{\beta_{i-1}^2}{\delta_{i-1}} \text{ si } i>1
            \end{cases}
    \end{equation}

    [BORRADOR] Como el metodo de Thomas, es un caso especial de factorización LU, se sabe que $ Q = L_n\Delta_nL_{n}^{t} $,
    escribamos $\Delta_n$ como $\sqrt\Delta_n\sqrt\Delta_n$ y sustituyendo en la expresion anterior se tiene que
    $ Q = L_n\sqrt\Delta_n\sqrt\Delta_nL_{n}^{t} $ y llamando $L$ a $L_n\sqrt\Delta_n$ y $L^t$ a $\sqrt\Delta_nL_{n}^{t}$
    obtenemos que $Q = LL^t$ con $L$ triangular inferior hallando asi su descomposicion de Cholesky.

    %Ej4.4
    \subsection{}

    %Ej4.5
    \subsection{}

    %Ej5
    \section{Pruebas Experimentales}
    Quinto problema

    %Ej5.1
    \subsection{}

    %Ej5.2
    \subsection{}

    %Ej5.3
    \subsection{}

    %Ej5.4
    \subsection{}

    
\end{document}
