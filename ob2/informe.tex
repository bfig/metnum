\documentclass{endm}
\usepackage{endmmacro}
\usepackage{graphicx}
\usepackage{amssymb,amsmath,latexsym}
\usepackage[varg]{pxfonts}
%%%%%% ENTER ADDITIONAL PACKAGES
%\usepackage{graphics}
\usepackage{pst-all}
\usepackage{graphicx}

\usepackage{amsfonts}
\usepackage{amssymb} % ADDED
\usepackage{times}
\usepackage{latexsym}
\usepackage{fancybox}
\usepackage{algorithm}
%\usepackage{algorithmic}
\usepackage{algorithmicx}
\usepackage{algpseudocode}
\usepackage{setspace}
\usepackage{courier}
\usepackage{verbatim}
\usepackage{hhline}
\usepackage{etex}
\usepackage{graphicx}
\usepackage{listings}
\usepackage{tikz}
\usetikzlibrary{calc,arrows,automata}
\usetikzlibrary{matrix,positioning,arrows,decorations.pathmorphing,shapes}
\usetikzlibrary{shapes,snakes}
\usepackage{graphicx}
%---------------
\usepackage{subfigure}
\usepackage{mathtools}
\usepackage{booktabs}
\usepackage{hyperref}
\makeatletter
\let\c@author\relax
\makeatother
\usepackage[backend=bibtex,style=alphabetic,
sorting=ynt
]{biblatex}
\addbibresource{sources.bib}
\lstdefinestyle{mystyle}{
    backgroundcolor=\color{backcolour},   
    commentstyle=\color{codegreen},
    keywordstyle=\color{magenta},
    numberstyle=\tiny\color{codegray},
    stringstyle=\color{codepurple},
    basicstyle=\ttfamily\footnotesize,
    breakatwhitespace=false,         
    breaklines=true,                 
    captionpos=b,                    
    keepspaces=true,                 
    numbers=left,                    
    numbersep=5pt,                  
    showspaces=false,                
    showstringspaces=false,
    showtabs=false,                  
    tabsize=2
}


\tolerance=1
\emergencystretch=\maxdimen
\hyphenpenalty=10000
\hbadness=10000


\floatname{algorithm}{Algorithm}


\def\lastname{Please list your Lastname here}

\begin{document}


\begin{frontmatter}

\title{Métodos Numéricos 2019 - Obligatorio 2}

\author{Bruno Figares (4391788-8),}
\author{Adrián Gioda (4954044-5),}
\author{Daniel Martinez (4462694-5),}
\author{Adriana Soucoff (3190794-8)}

\address{Instituto de Matem\'atica y Estad\'istica\\ Facultad de Ingenier\'ia. Universidad de la Rep\'ublica\\ Montevideo, Uruguay}

\end{frontmatter}


\section{Mínimos Cuadrados}
%Ej2.1
\subsection{Transformación a PMCL}
El modelo de ambas funciones tiene la forma $y = cx^{-p}$. Esta puede transformarse en una relación
lineal aplicando un logaritmo a ambos lados de la expresión de esta forma, $log(y) = log(cx^{-p})$.
Por propiedades de los logaritmos, la expresión se reduce a $log(y) = log(c) + (-p)log(x)$.
Aplicando cambios de variable apropiados, se obtiene la relación lineal $Y = c_2 + c_1X$.

Se busca el vector $C$ de coeficientes que minimizá $||AC - Y||^2_2$,
con $A = \begin{pmatrix}X_1 & 1 \\ \vdots & \vdots \\ X_m & 1 \end{pmatrix}$.
Para esto se deben resolver las ecuaciones normales $A^tAC = A^tY$.

Resolución de las ecuaciones normales:
desarrollamos para el caso de $g_{1}$ es análogo para $g_{2}$ sustituyendo $c$ por $d$ y $p$ por $q$ 

$Y =log(y)$ , $X =log(x)$ , $c_1= log(c)$, $c_2= -p$ \\
siendo $n$ el largo del vector $X$ tenemos las ecuaciones normales
\begin{equation}
\begin{cases}
\sum Y = n * c_1 + c_2 * \sum X \\
\sum X * Y = c_1 * \sum X + c_2 * \sum X^2
\end{cases}
\end{equation}  \\
Siendo C=\begin{pmatrix}
          c_1 & \\
          c_2
         \end{pmatrix} \\
         
Resolvemos : $ C=(A^t * A) \setminus  (A^t * Y)$ \\
Finalmente obtenemos:  $c= exp(c_1)$, $p = -c_2$ \\


Aplicación de descomposición QR:
Debido a que el calculo de $A^tA$ esta mal condicionado, se aplica la descomposición QR de $A$.
Teniendo que $A \in \mathcal{M}_{m \times n}$, con $m > n$ tiene rango completo (sus columnas son LI),
por el teorema 4.3.1 de los apuntes se tiene que existen matrices $Q \in \mathcal{M}_{m \times m}$,
$R \in \mathcal{M}_{m \times n}$ tales que $A = QR$. Con $Q$ ortogonal
y $R$ triangular superior de la forma \begin{pmatrix} R_1 \\ 0 \end{pmatrix} con $R_1 \in \mathcal{M}_{n \times n}$.

Sustituyendo $A$ por $QR$, el problema de minimizacion se torna $min ||QRC - Y||^2_2$.
Separando a $Q$ en dos partes $[Q_1 Q_2]$ tal que $Q_1 \in \mathcal{M}_{m \times n}$ y
$Q_2 \in \mathcal{M}_{m \times (m-n)}$ y operando, se llega al problema de minimizacion $min ||R_1C - Q_1^tY||^2_2$
cuya solución proviene del sistema de ecuaciones $R_1C = Q_1^tY$, que como $R_1$ es triangular superior se resuelve con
sustitución hacia atrás.

% Falta mostrar las matrices A, Y, Q1, R1 y la solucion C. Ademas hay que poner las graficas. %



%Ej2.2
\subsection{Resolución por Gauss-Newton}


%Ej2.3
\subsection{Comparación de los Métodos}


%Ej3
\section{Ecuaciones Diferenciales}
Se busca resolver el siguiente problema de valores iniciales:
\begin{equation*}
    (PVI):\begin{cases}
        y'(x) = -g_1(x)y + g_2(x) \\
        y(1/2) = 0
    \end{cases}
\end{equation*}

%Ej3.1
\subsection{Resolución Analítica}
Se resuelve analíticamente en primer lugar a fin de poder evaluar la solución provista por distintos métodos numéricos.
En la parte 2 se vio que $g_1(x) = cx^{-2}$ y $g_2(x) = dx^{-3}$ con $c,d \in \mathbb{R}$.
La ecuación diferencial es entonces:
\begin{equation}
    y' + \frac{c}{x^2}y = \frac{d}{x^3}
\end{equation}
Solución de la homogénea:
\begin{cases}
    y' + \frac{c}{x^2}y = 0 \\
    y_h = exp(-\int cx^{-2}dx) \\
    y_h = ke^{c/x}
\end{cases}
Variación de constantes:
Se escribe la función $y$ como $y = k(x)e^{c/x}$. La idea es obtener la expresión de $k(x)$.
Se deriva $y$: $y' = k'(x)e^{c/x} + k(x)\left(-\frac{c}{x^2}e^{c/x}\right)$
Sustituyendo en la EDO, se cancelan dos términos y se despeja $k'(x)$. Integrando la expresión resultante se obtiene $k(x)$.
\begin{equation}
    k(x) = \int \frac{d}{x^3}e^{-c/x}dx
\end{equation}
Aplicando integración por partes con $u = \frac{d}{x}$, $du = -\frac{d}{x^2}$, $v = e^{-c/x}$ y $dv = \frac{c}{x^2}e^{c/x}$ se llega a:
\begin{equation}
    k(x) = \frac{1}{c}\left( \frac{d}{x}e^{-c/x} + C_1 + \frac{d}{c}\int e^{-c/x}\frac{c}{x^2}dx \right)
\end{equation}
Aplicando integración por sustitución con la misma $v$ y $dv$ se tiene que:
\begin{equation}
    k(x) = \frac{1}{c}\left( \frac{d}{x}e^{-c/x} + C_1 + \frac{d}{c}e^{-c/x} + C_2 \right) =
    \frac{d}{c}e^{-c/x}\left( \frac{1}{x} + \frac{1}{c} \right) + k
\end{equation}
Finalmente se llega a la expresión de $y$:
\begin{equation}
    y(x) = \frac{d}{c^2} + \frac{d}{cx}+ ke^{c/x}
\end{equation}
Usando la condición inicial $y(1/2) = 0$ se halla que $k = -\frac{d}{c}\left( \frac{1}{c} + 2 \right)e^{-2c}$.

%Ej3.2
\subsection{Métodos de Euler}


%Ej3.3
\subsection{Implementación de los Métodos de Eueler}


%Ej3.4
\subsection{Método de Runge-Kutta}


%Ej3.5
\subsection{Comparación de los Resultados}


%Ej4
\section{Interpolación}

%Ej4.1
\subsection{Interpolación Lineal a Trozos}


%Ej4.2
\subsection{Interpolación con Splines Cúbicos}


%Ej4.3
\subsection{Comparación de los Resultados}


\clearpage
\printbibliography
\end{document}\grid
