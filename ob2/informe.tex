\documentclass{endm}
\usepackage{endmmacro}
\usepackage{graphicx}
\usepackage{amssymb,amsmath,latexsym}
\usepackage[varg]{pxfonts}
%%%%%% ENTER ADDITIONAL PACKAGES
%\usepackage{graphics}
\usepackage{pst-all}
\usepackage{graphicx}

\usepackage{amsfonts}
\usepackage{amssymb} % ADDED
\usepackage{times}
\usepackage{latexsym}
\usepackage{fancybox}
\usepackage{algorithm}
%\usepackage{algorithmic}
\usepackage{algorithmicx}
\usepackage{algpseudocode}
\usepackage{setspace}
\usepackage{courier}
\usepackage{verbatim}
\usepackage{hhline}
\usepackage{etex}
\usepackage{graphicx}
\usepackage{listings}
\usepackage{tikz}
\usetikzlibrary{calc,arrows,automata}
\usetikzlibrary{matrix,positioning,arrows,decorations.pathmorphing,shapes}
\usetikzlibrary{shapes,snakes}
\usepackage{graphicx}
%---------------
\usepackage{subfigure}
\usepackage{mathtools}
\usepackage{booktabs}
\usepackage{hyperref}
\makeatletter
\let\c@author\relax
\makeatother
\usepackage[backend=bibtex,style=alphabetic,
sorting=ynt
]{biblatex}
\addbibresource{sources.bib}
\lstdefinestyle{mystyle}{
    backgroundcolor=\color{backcolour},   
    commentstyle=\color{codegreen},
    keywordstyle=\color{magenta},
    numberstyle=\tiny\color{codegray},
    stringstyle=\color{codepurple},
    basicstyle=\ttfamily\footnotesize,
    breakatwhitespace=false,         
    breaklines=true,                 
    captionpos=b,                    
    keepspaces=true,                 
    numbers=left,                    
    numbersep=5pt,                  
    showspaces=false,                
    showstringspaces=false,
    showtabs=false,                  
    tabsize=2
}


\tolerance=1
\emergencystretch=\maxdimen
\hyphenpenalty=10000
\hbadness=10000


\floatname{algorithm}{Algorithm}


\def\lastname{Please list your Lastname here}

\begin{document}


\begin{frontmatter}

\title{Métodos Numéricos 2019 - Obligatorio 2}

\author{Bruno Figares (4391788-8),}
\author{Adrián Gioda (4954044-5),}
\author{Daniel Martinez (4462694-5),}
\author{Adriana Soucoff (3190794-8)}

\address{Instituto de Matem\'atica y Estad\'istica\\ Facultad de Ingenier\'ia. Universidad de la Rep\'ublica\\ Montevideo, Uruguay}

\end{frontmatter}


\section{Mínimos Cuadrados}
%Ej2.1
\subsection{Transformación a PMCL}
El modelo de ambas funciones tiene la forma $y = cx^{-p}$. Esta puede transformarse en una relación
lineal aplicando un logaritmo a ambos lados de la expresión de esta forma, $log(y) = log(cx^{-p})$.
Por propiedades de los logaritmos, la expresión se reduce a $log(y) = log(c) + (-p)log(x)$.
Aplicando cambios de variable apropiados, se obtiene la relación lineal $Y = c_2 + c_1X$.

Se busca el vector $C$ de coeficientes que minimizá $||AC - Y||^2_2$,
con $A = \begin{pmatrix}X_1 & 1 \\ \vdots & \vdots \\ X_m & 1 \end{pmatrix}$.
Para esto se deben resolver las ecuaciones normales $A^tAC = A^tY$.

Resolución de las ecuaciones normales:
desarrollamos para el caso de $g_{1}$ es análogo para $g_{2}$ sustituyendo $c$ por $d$ y $p$ por $q$ 

$Y =log(y)$ , $X =log(x)$ , $c_1= log(c)$, $c_2= -p$ \\
siendo $n$ el largo del vector $X$ tenemos las ecuaciones normales
\begin{equation}
\begin{cases}
\sum Y = n * c_1 + c_2 * \sum X \\
\sum X * Y = c_1 * \sum X + c_2 * \sum X^2
\end{cases}
\end{equation}  \\
Siendo \begin{equation*}C=\begin{pmatrix}
          c_1 & \\
          c_2
         \end{pmatrix} \\
\end{equation*}
Resolvemos : $ C=(A^t * A) \setminus  (A^t * Y)$ \\
Finalmente obtenemos:  $c= exp(c_1)$, $p = -c_2$ \\


Aplicación de descomposición QR:
Debido a que el calculo de $A^tA$ esta mal condicionado, se aplica la descomposición QR de $A$.
Teniendo que $A \in \mathcal{M}_{m \times n}$, con $m > n$ tiene rango completo (sus columnas son LI),
por el teorema 4.3.1 de los apuntes se tiene que existen matrices $Q \in \mathcal{M}_{m \times m}$,
$R \in \mathcal{M}_{m \times n}$ tales que $A = QR$. Con $Q$ ortogonal
y $R$ triangular superior de la forma  \begin{equation*} \begin{pmatrix} R_1 \\ 0 \end{pmatrix}\end{equation*} con $R_1 \in \mathcal{M}_{n \times n}$.

Sustituyendo $A$ por $QR$, el problema de minimizacion se torna $min ||QRC - Y||^2_2$.
Separando a $Q$ en dos partes $[Q_1 Q_2]$ tal que $Q_1 \in \mathcal{M}_{m \times n}$ y
$Q_2 \in \mathcal{M}_{m \times (m-n)}$ y operando, se llega al problema de minimizacion $min ||R_1C - Q_1^tY||^2_2$
cuya solución proviene del sistema de ecuaciones $R_1C = Q_1^tY$, que como $R_1$ es triangular superior se resuelve con
sustitución hacia atrás.

% Falta mostrar las matrices A, Y, Q1, R1 y la solucion C. Ademas hay que poner las graficas. %



%Ej2.2
\subsection{Resolución por Gauss-Newton}
Otra manera de resolver el problema sin tener que utilizar logaritmos es a través del uso de los polinomios de taylor lineales. Para esto, para ajustar los puntos $X_i,Y_i$, con $ i \in \{1,..,n\}$

\begin{align*}
&F(c,p) =||X_{c,p} - Y||^2_2 \\
&X_{c,p} = c*X_i^p \forall i \in \{1,..n\}
\end{align*}

Con este problema nuevo, asumimos que estamos cerca de una solución y linealizamos el problema a través del polinomio de taylor de orden 1, con la posibilidad de obtener una solución con un error acotado por el cuadrado de la distancia de la solución inicial $k$
\begin{equation*}
F_k = ||  F(k) + \Delta F(k)\Delta_{c,p} - Y + o(\Delta_{c,p}^2)||^2_2\\
\end{equation*}
Haciendo los siguientes cambios de variable obtenemos una secuencia de PMCL que podemos resolver y encontrar soluciones progresivamente mejores
\begin{align*}
&F_{k_i} \equiv || F(k_i) + \Delta F(k_i)\Delta_{c,p} - Y||^2_2\\
&Y_{k_i} = Y - F(k_i) \\
&k_{i+1} = k_i + \min_{\Delta_{c,p}} || \Delta F(k_i)\Delta_{c,p} - Y_{k_i} ||^2_2
\end{align*}

%Ej2.3
\subsection{Comparación de los Métodos}
Los métodos todos llegan a una solución similar, si bien en este caso la solución funciona bien, en otros casos puede que la transformación cambie sustancialmente la precisión del error ya que el logaritmo afecta de manera más profunda a ciertos puntos que a otros. Si los puntos de la gráfica ajustan perfectamente a una curva con los valores, podemos llegar al mismo número con ambas soluciones, pero sino pasa esto (todos los casos que realmente nos importan donde estamos encontrando una mejor aproximación) hay que comparar la fórmula de los errores de las derivadas parciales de las coordenadas de Y frente a la función de ajuste. Se utiliza una simplificación de la fórmula del error asumiendo que la función de ajuste se mantiene estática con el mejor ajuste previo para el punto para mantener el análisis simple.

\begin{align*}
&Err(Y_i)= (X_i-Y_i)^2\\
&Err_{log}(Y_i)= (log(X_i)-log(Y_i))^2\\
\end{align*}
entonces
\begin{align*}
&\frac{\delta Err}{\delta Y_i} = 2Y_i - 2X_i\\
&\frac{\delta Err_{log}}{\delta Y_i} = 2\frac{log(Y_i)-log(X_i)}{Y_i}\\
\end{align*}

Conforme los errores sean mayores, el PMCL simplificado va a dar soluciones que valoren menos el error puntual en proporción en el caso de los logaritmos. Por otro lado, es mucho más fácil de calcular con logaritmos. Una cosa que se podría hacer para un problema genérico del estilo es usar primero el problema simplificado con logaritmos y después usar Newton Raphson para llegar a una solución al problema inicial.

%Ej3
\section{Ecuaciones Diferenciales}
Se busca resolver el siguiente problema de valores iniciales:
\begin{equation*}
    (PVI):\begin{cases}
        y'(x) = -g_1(x)y + g_2(x) \\
        y(1/2) = 0
    \end{cases}
\end{equation*}

%Ej3.1
\subsection{Resolución Analítica}
Se resuelve analíticamente en primer lugar a fin de poder evaluar la solución provista por distintos métodos numéricos.
En la parte 2 se vio que $g_1(x) = cx^{-2}$ y $g_2(x) = dx^{-3}$ con $c,d \in \mathbb{R}$.
La ecuación diferencial es entonces:
\begin{equation}
    y' + \frac{c}{x^2}y = \frac{d}{x^3}
\end{equation}
Solución de la homogénea:
 \begin{equation*}
\begin{cases}
    y' + \frac{c}{x^2}y = 0 \\
    y_h = exp(-\int cx^{-2}dx) \\
    y_h = ke^{c/x}
\end{cases}
\end{equation*}
Variación de constantes:
Se escribe la función $y$ como $y = k(x)e^{c/x}$. La idea es obtener la expresión de $k(x)$.
Se deriva $y$: $y' = k'(x)e^{c/x} + k(x)\left(-\frac{c}{x^2}e^{c/x}\right)$
Sustituyendo en la EDO, se cancelan dos términos y se despeja $k'(x)$. Integrando la expresión resultante se obtiene $k(x)$.
\begin{equation}
    k(x) = \int \frac{d}{x^3}e^{-c/x}dx
\end{equation}
Aplicando integración por partes con $u = \frac{d}{x}$, $du = -\frac{d}{x^2}$, $v = e^{-c/x}$ y $dv = \frac{c}{x^2}e^{c/x}$ se llega a:
\begin{equation}
    k(x) = \frac{1}{c}\left( \frac{d}{x}e^{-c/x} + C_1 + \frac{d}{c}\int e^{-c/x}\frac{c}{x^2}dx \right)
\end{equation}
Aplicando integración por sustitución con la misma $v$ y $dv$ se tiene que:
\begin{equation}
    k(x) = \frac{1}{c}\left( \frac{d}{x}e^{-c/x} + C_1 + \frac{d}{c}e^{-c/x} + C_2 \right) =
    \frac{d}{c}e^{-c/x}\left( \frac{1}{x} + \frac{1}{c} \right) + k
\end{equation}
Finalmente se llega a la expresión de $y$:
\begin{equation}
    y(x) = \frac{d}{c^2} + \frac{d}{cx}+ ke^{c/x}
\end{equation}
Usando la condición inicial $y(1/2) = 0$ se halla que $k = -\frac{d}{c}\left( \frac{1}{c} + 2 \right)e^{-2c}$.

%Ej3.2
\subsection{Métodos de Euler}
% Euler hacia adelante %

El método de Euler hacia atrás estima la derivada en el punto $x_{n+1}$ imponiendo $y'(x_{n+1}) = \frac{y_{n+1} - y_n}{h} = f(x_{n+1},y_{n+1})$.
Entonces se tiene la iteración:
\begin{cases}
    y_{k+1} = y_k + hf(x_{k+1},y_{k+1}) \\
    y_0 = y_{inicial}
\end{cases}
Debido a que $y_{k+1}$ aparece en ambos lados de la ecuación, este es un método implícito. Sin embargo, si
$f$ es lineal en $y$, se puede obtener una formulación explicita.

En este caso, la EDO a resolver es $y' = f(x,y) = \frac{d}{x^3} - \frac{c}{x^2}y$. Dado que $f$ es lineal en $y$, se desarrolla
la formulación explicita para $y_{k+1}$.
\begin{align*}
    y_{k+1} &= y_k + hf(x_{k+1},y_{k+1}) \\
    &= y_k + h\left( \frac{d}{x_{k+1}^3} - \frac{c}{x_{k+1}^2}y_{k+1} \right)
\end{align*}
Despejando $y_{k+1}$ se obtiene:
\begin{equation}
    y_{k+1} = \frac{y_k + \frac{hd}{x_{k+1}^3}}{1 + \frac{hc}{x_{k+1}^2}}
    % otra forma menos compacta: %
    %y_{k+1} = \left( \frac{1}{1 + \frac{hc}{x_{k+1}^2}} \right) \left( y_k + \frac{hd}{x_{k+1}^3}\right)%
\end{equation}.

%Ej3.3
\subsection{Implementación de los Métodos de Eueler}


%Ej3.4
\subsection{Método de Runge-Kutta}
Los métodos de Runge-Kuta calculan el valor de un punto $y_{n+1}$ a partir del punto anterior $y_n$ y de
un promedio ponderado de las derivadas en distintos puntos en el intervalo $[x_n, x_{n+1}]$.
El usado para esta sección es el implementado por la función ode45 de Octave.
Este es un método de 6 etapas de orden 5 y funciona de la siguiente manera:
Empezando en el punto $(x_n,y_n)$ con pendiente $s_1 = y' = f(x_n,y_n)$ y un paso $h$ de tamaño apropiado,
se busca computar $y_{n+1}$ en el punto $x_{n+1} + h$.
Usando "Euler hacia adelante" con la pendiente $s_1$ y una fracción del paso $h$ se obtiene un segundo punto
con el cual se calcula la pendiente $s_2$. Este proceso se repite hasta conseguir $s_6$.
Mediante un promedio ponderado entre las 6 pendientes se calcula $y_{n+1}$
Se calcula una ultima pendiente $s_7$ en $(x_{n+1},y_{n+1})$ que servirá para calcular una estimación
del error que sera usado para decidir el paso $h$ para el calculo del siguiente punto.

La función ode45 produce una sucesión de puntos $(x_n,y_n)$ que parecen ser demasiado espaciados como se vera
en las gráficas de la sección siguiente. Debido a esto, puede necesitarse un interpolador para poder crear una
gráfica mas suave en Octave. Esto se mostrara en la sección 3.

%Ej3.5
\subsection{Comparación de los Resultados}


%Ej4
\section{Interpolación}
 j4.1
\subsection{Interpolación Lineal a Trozos}


%Ej4.2
\subsection{Interpolación con Splines Cúbicos}


%Ej4.3
\subsection{Comparación de los Resultados}


\clearpage
\printbibliography
\end{document}\grid
